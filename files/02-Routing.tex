\chapter{Routing}

\section{Allgemeines}

Routing bezeichnet in der Telekommunikation das Festlegen von Wegen für Nachrichtenströme in Rechnernetzen. Dabei ist bei paketvermittelten Netzen zwischen Forwarding und Routing zu unterscheiden. Forwarding beschreibt einen Entscheidungsprozess eines Netzwerkknotens, das Routing hingegen, bestimmt den gesamten Weg eines Nachrichtenstroms durch das Netzwerk. Trotzdem werden diese beide Begriffe oftmals als 'Routing' bezeichnet, in diesem Fall bezeichnet es die ganze allgemeine Übermittlung von Nachrichten (Paketen) über Netzwerke.\\\\
Beim paketvermittelten Routing, wie es z. B. im Internet stattfindet, wird dafür gesorgt, dass logisch adressierte Datenpakete aus dem Ursprungsnetz herauskommen und in Richtung ihres Zielnetzes weitergeleitet werden. Routing ist die Basis des Internets – ohne Routing würde das Internet nicht existieren, und alle Netze wären autonom. Die Datenpakete können dabei viele verschiedene Zwischennetze auf dem Weg zu ihrem Ziel passieren. Im Internet wird das Routing (üblicherweise) auf der IP-Schicht durchgeführt.
Router arbeiten auf Schicht 3 (Vermittlungsschicht/Network Layer) des OSI-Referenzmodells. Ein Router besitzt mehrere Schnittstellen (engl. Interfaces), über die Netze erreichbar sind. Diese Schnittstellen können auch virtuell sein. Beim Eintreffen von Datenpaketen muss ein Router anhand der OSI-Schicht-3-Zieladresse (z. B. IP-Adresse) den besten Weg zum Ziel und damit die passende Schnittstelle bestimmen, über welche die Daten weiterzuleiten sind. Dazu bedient er sich einer lokal vorhandenen Routingtabelle, die angibt, über welchen Anschluss des Routers oder welche Zwischenstation welches Netz erreichbar ist.\\\\
Router können Wege auf drei verschiedene Arten lernen und mit diesem Wissen dann die Routingtabelleneinträge erzeugen:

\begin{itemize}
\item \textbf{direkt verbundene Netze}: Sie werden automatisch in eine Routingtabelle übernommen, wenn ein Interface mit einer IP-Adresse konfiguriert wird.
\item \textbf{statische Routen}: Diese Wege werden durch einen Administrator eingetragen. Sie dienen zum einen der Sicherheit, sind andererseits aber nur verwaltbar, wenn ihre Zahl begrenzt ist, d. h. die Skalierbarkeit ist für diese Methode ein limitierender Faktor.
\item \textbf{dynamische Routen}: In diesem Fall lernen Router erreichbare Netze durch ein Routingprotokoll, das Informationen über das Netzwerk und seine Teilnehmer sammelt und an die Mitglieder verteilt.
\end{itemize}

Die Routingtabelle ist in ihrer Funktion einem Adressbuch vergleichbar, in dem nachgeschlagen wird, ob eine Ziel-IP-Adresse bekannt ist, d. h. ein Weg zu diesem Netz existiert. Da ein Router nicht für alle IP-Adressen darauf eine Antwort weiß, muss es eine Standardvorgabe geben. Da Routingtabellen bei den meisten Systemen nach der Genauigkeit sortiert werden, also zuerst spezifische Einträge und später weniger spezifische, kommt die Default-Route, als unspezifische, am Ende und wird für alle Ziele benutzt, die über keinen besser passenden, spezifischeren Eintrag in der Routingtabelle verfügen.\\\\

Einige Router beherrschen auch ein sogenanntes Policy Based Routing; dabei wird die Routingentscheidung nicht nur auf Basis der Zieladresse (Layer-3) getroffen, sondern es werden zusätzlich andere Angaben berücksichtigt, beispielsweise die Quelladresse, Qualitätsanforderungen oder Parameter aus höheren Schichten wie TCP oder UDP. So können dann zum Beispiel Pakete, die HTTP (Web) transportieren, einen anderen Weg nehmen als Pakete mit SMTP-Inhalten (Mail).
