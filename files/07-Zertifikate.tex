\chapter{Zertifikate}

Digitale Zertifikate ist ein digitaler Datensatz (=Datenformat), der durch kryptographische Verfahren Authenzität und Integrität bietet. Diese Zeritikate beinhalten kryptographisch geschütze Informationen über die Echtheit von Objekten, Personen, Schlüssel usw.\\
Ein weit verbreitetes Zertifikatstandard ist der X.509-Standard, der heute z.B. für HTTPS verwendet wird. Dieses X.509-Zertifikat hat die Eigenschaft die Indentität des Inhabers und andere Informationen (Öffentlicher Schlüssel, Ablaufdatum des Zertifikates, ...) zu bestätigen. Der X.509-Standard unterscheidet die gewöhnlichen Zertifikate von den Root-Zertifikaten. Root-Zertifikate sind in der Lage, andere Root-Zertifikate oder gewöhnliche zu unterschreiben. Nun muss der Client nur das Root-Zertifikat vertrauen und vertraut automatisch jedes Zertifikat, dass von ihr unterschrieben worden ist. Die Idee ist, dass man am Gerät nur ein Zertikat draufspielt und die Zertifizierungsstelle (Inhaber dieses Zertifikates) Inhaber von anderen Seiten zertifizieren können.
