\chapter{Netzsicherheit (wiki copy'n'paste)}
\section{Was ist Netzsicherheit}
Netzwerksicherheit (auch Netzsicherheit) ist kein einzelner feststehender Begriff, sondern umfasst alle Maßnahmen zur Planung, Ausführung und Überwachung der Sicherheit in Netzwerken. Diese Maßnahmen sind keinesfalls nur technischer Natur, sondern beziehen sich auch auf die Organisation (z. B. Richtlinien, in denen geregelt wird, was die Betreiber des Netzwerkes dürfen sollen), den Betrieb (Wie kann ich Sicherheit im Netzwerk in der Praxis anwenden, ohne gleichzeitig den Ablauf des Betriebs zu stören?) und schließlich auch auf das Recht (Welche Maßnahmen dürfen eingesetzt werden?).
\section{Aspekte der Netzwerksicherheit}
\subsection{Informationssicherheit}
Als Informationssicherheit bezeichnet man Eigenschaften von informationsverarbeitenden und -lagernden (technischen oder nicht-technischen) Systemen, die die Schutzziele Vertraulichkeit, Verfügbarkeit und Integrität sicherstellen. Informationssicherheit dient dem Schutz vor Gefahren bzw. Bedrohungen, der Vermeidung von wirtschaftlichen Schäden und der Minimierung von Risiken.\\

Information (oder Daten) sind schützenswerte Güter. Der Zugriff auf diese sollte beschränkt und kontrolliert sein. Nur autorisierte Benutzer oder Programme dürfen auf die Information zugreifen. Schutzziele werden zum Erreichen bzw. Einhalten der Informationssicherheit und damit zum Schutz der Daten vor beabsichtigten Angriffen von IT-Systemen definiert:

\begin{itemize}
\item Vertraulichkeit (englisch: confidentiality): Daten dürfen lediglich von autorisierten Benutzern gelesen bzw. modifiziert werden, dies gilt sowohl beim Zugriff auf gespeicherte Daten, wie auch während der Datenübertragung.
\item Integrität (englisch: integrity): Daten dürfen nicht unbemerkt verändert werden. Alle Änderungen müssen nachvollziehbar sein.
\item Verfügbarkeit (englisch: availability): Verhinderung von Systemausfällen; der Zugriff auf Daten muss innerhalb eines vereinbarten Zeitrahmens gewährleistet sein.
\item Authentizität (englisch: authenticity) bezeichnet die Eigenschaften der Echtheit, Überprüfbarkeit und Vertrauenswürdigkeit eines Objekts.
\item Verbindlichkeit/Nichtabstreitbarkeit (englisch: non repudiation): Sie erfordert, dass kein unzulässiges Abstreiten durchgeführter Handlungen möglich ist. Sie ist unter anderem wichtig beim elektronischen Abschluss von Verträgen. Erreichbar ist sie beispielsweise durch elektronische Signaturen.
\item Zurechenbarkeit (englisch: accountability): Eine durchgeführte Handlung kann einem Kommunikationspartner eindeutig zugeordnet werden.
\item Anonymität
\end{itemize}

\subsection{AAA}
Die drei A's des Trible A Systems stehen für Authentication, Authorization und Accounting. 